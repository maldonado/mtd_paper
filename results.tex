The goal of our study is to better understand technical debt found in source code comments. To do so, we first manually analyze the comments identifying technical debt comments in the selected projects (RQ1). Then, we classify the comments into technical debt types accordingly with their nature (RQ2). In the remainder of this section we detail the motivation, approach and results for each of our research questions.  

\vspace{3mm}
\noindent\rqi
\vspace{3mm}

\noindent\textbf{Motivation:} Understand the frequency of this kind of comments using projects from different domains and size. Analyzing source code comments open several opportunities to get insights obtained from human elicitation. These comments can point the direction to tools and future maintenance work as they were analyzed by the authors of the the program. 

\vspace{1mm}
\noindent\textbf{Approach:} 

\vspace{1mm}
\noindent\textbf{Results:} Put some examples of the found technical debt comments, explain that for some of them the classification was clear but for others were necessary more investigation looking into the source code to understand the context. explain the numbers found in the projects. create a table to present this.

\vspace{3mm}
\noindent\rqii
\vspace{3mm}

\noindent\textbf{Motivation:} Understand the different types of technical debt found in comments can 

\vspace{1mm}
\noindent\textbf{Approach:}

\vspace{1mm}
\noindent\textbf{Results:}  


the number of requirement debt shows the maturity of the project by the perspective of the developers. a lot of opportunities to improve features of the project can mean that the project still in a immature phase. or that there still a lot of room to improvement. it means that is not in the more stable state that it can be. or that the authors were expecting. when the project is new there is a lot of thing that the developers still want to develop. in more stable projects the predominance is design debits not requirements (implementation). the number of features that one product offers can play a role in the implementation debt as well, as the role of the application is more complex one will find more things to implement as well. like argo has many features, and many users, this can provide the developers with the necessary direction to evolve the project . whereas columba has just a few developers so there are a lot of things that need to be implemented as well, the number of users can put this number up or down , a lot of users can force the developers to better implement the software. tell about the the difference between very large softwares against very specialist systems. 


contributions with this paper:

The definition of the different types of self-admitted technical debt found in source code comments.
The quantification of each of these types of self-admitted technical debt.
Making the dataset used in this study available to the software engineering community.  

\begin{table*}[!hbt]
      \begin{center}
            \caption{Project Details}
            \label{tab:project_details}
            \begin{tabular}{l| c c c c }
            \toprule
            \textbf{Project}   & \textbf{Release}          & \textbf{Analyzed comments}     & \textbf{Self-admitted TD comments} & \textbf{Percentage} \\ \midrule 
              Apache Ant       & 1.7.0                     & 4,140                          & 134                                & 3.2  \\                                   
              Apache Jmeter    & 2.10                      & 8,163                          & 375                                & 4.6  \\                                   
              ArgoUML          & 0.34                      & 9,788                          & 1,653                              & 16.8 \\                                   
              JFreeChart       & 1.0.19                    & 4,433                          & 219                                & 4.9  \\ \bottomrule
            \end{tabular}
      \end{center}
\end{table*}


\begin{table*}[!hbt]
      \begin{center}
            \caption{Apache Ant Self-Admitted Technical Debt distribution}
            \label{tab:ant_td_details}
            \begin{tabular}{l| c c }
            \toprule
            \textbf{Type}   & \textbf{\# of comments}     & \textbf{Percentage}  \\ \midrule 
             Defect          & 13             & 9.70\\     
             Design          & 95             & 70.89\\    
             Documentation   & 0              & 0.00\\      
             Implementation  & 16             & 11.94\\    
             Test            & 10             & 7.46\\  \bottomrule                                   
            \end{tabular}
      \end{center}
\end{table*}

\begin{table*}[!hbt]
      \begin{center}
            \caption{Apache Jmeter Self-Admitted Technical Debt distribution}
            \label{tab:jmeter_td_details}
            \begin{tabular}{l| c c }
            \toprule
            \textbf{Type}    & \textbf{\# of comments}     & \textbf{Percentage}  \\ \midrule 
             Defect          & 22              & 5.86\\     
             Design          & 316             & 84.26\\    
             Documentation   & 3               & 0.8\\      
             Implementation  & 22              & 5.86\\    
             Test            & 12              & 3.2\\  \bottomrule                                   
            \end{tabular}
      \end{center}
\end{table*}

\begin{table*}[!hbt]
      \begin{center}
            \caption{ArgoUml Self-Admitted Technical Debt distribution}
            \label{tab:argo_td_details}
            \begin{tabular}{l| c c }
            \toprule
            \textbf{Type}   & \textbf{\# of comments}     & \textbf{Percentage}  \\ \midrule 
             Defect          &  127            &7.68 \\     
             Design          &  801            &48.45 \\    
             Documentation   &  30             &1.81 \\      
             Implementation  &  651            &39.38 \\    
             Test            &  44             &2.66 \\  \bottomrule                                   
            \end{tabular}
      \end{center}
\end{table*}

\begin{table*}[!hbt]
      \begin{center}
            \caption{JFreechart Self-Admitted Technical Debt distribution}
            \label{tab:jfreechart_td_details}
            \begin{tabular}{l| c c }
            \toprule
            \textbf{Type}   & \textbf{\# of comments}     & \textbf{Percentage}  \\ \midrule 
             Defect          &  9              &4.10 \\     
             Design          &  184            &84.01 \\    
             Documentation   &  0              & 0.0\\      
             Implementation  &  25             & 11.41\\    
             Test            &  1              & 0.45\\  \bottomrule                                   
            \end{tabular}
      \end{center}
\end{table*}

