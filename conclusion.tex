The term technical debt is being used for practitioners and researchers in the software engineer community to express shortcuts and workarounds employed in software projects. These shortcuts will most often impact the maintainability of the project hindering the development if not addressed properly.

Our work explore specifically self-admitted technical debt, that is the technical debt deliberately introduced by the developers and reported by the same trough source code comments.

In our study we analyzed the comments of 5 open source projects which are Apache Ant, Apache Jmeter, ArgoUml , Columba and JFreeChart. These projects are considered well commented and they belong to different application domains. We used them to understand the characteristics of self-admitted technical debt types creating a rich dataset with more than 33,093 classified comments.

In our approach we present 4 simple filtering techniques that made feasible the hard task to manually classify the comments of a project. These filtering techniques significantly reduced the number of comments by 81\% of the total 166,756 comments, allowing the classification to be focused on the most applicable and insightful comments that was 33,093 comments in total. 

We find that self-admitted technical debt can be classified into the following types: design debt, defect debt, documentation debt, requirement debt and test debt. We also provide concrete examples of each one of the mentioned types and the rationale to classify them as it was.  

We also find that the majority of the self-admitted technical debt comments are of the design debt type. It ranged from 42\% to 84\% across the projects. The second most frequent type was requirement debt that ranged from 5\% to 45\%. Based on this result, we can say that the self-admitted technical debt types that developers cares the most are related with the design of the program and implementation of future features, outranking documentation and test debt.

Other contribution of our study is that we make public available the resulting dataset of our classification. We hope that this will encourage future research in the area of self-admitted technical debt as, to the best of our knowledge, this is the first dataset of this kind, and we also think that the information provided by this dataset can be a cornerstone for more advanced techniques as natural language processing.   

In a future work we plan to improve the current classification adding more projects to it. With a richer dataset we expect that more patterns and characteristics of the self-admitted technical types will be retrieved. We also plan to use this database to mine unique sequential patterns, an advanced technique of natural language processing that is a good fit to this kind of data. 
