The term technical debt are being used for practitioners and researchers in the software engineer community to express shortcuts and workarounds employed in software projects. These shortcuts will most often impact the maintainability of the project hindering the development if not addressed properly.

Our work explore specifically self-admitted technical debt, that is the technical debt deliberately introduced by the developers and reported by the same trough source code comments.

In our study we analyzed the comments of 5 open source projects which are Apache Ant, Apache Jmeter, ArgoUml , Columba and JFreeChart. These projects are considered well commented and they belong to different application domains. We used them to understand the characteristics of self-admitted technical debt types creating a rich dataset with more than 33,093 classified comments.

In our approach we present 4 simple filtering techniques that made feasible the hard task to manually classify the comments of a project. These filtering techniques significantly reduced the number of comments by 81\% of the total 166,756 comments, allowing the classification to be focused on the most applicable and insightful comments that was 33,093 comments in total. 

We find that self-admitted technical debt can be classified into the following types: design debt, defect debt, documentation debt, requirement debt and test debt. We also provide concrete examples of each one of the mentioned types and our reasoning to classify them.  

\todo{keep going from here}

% In a future work we plan to include in our analyze the severity of the issues. In our current analysis we do not take this factor in consideration. This will lead us further in the understanding of Defect Debt characterization, and prevent us of relying just on the assumption that only what is import get fixed. As a matter of fact, more research in this subject can provide empirical evidence for proving or rejecting this assumption.