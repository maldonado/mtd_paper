\noindent\textbf{Internal validity} consider the relationship between theory and observation, in case the measured variables do not measure the actual factors. To classify the source code comments we heavily depended on manual process due the fact that comments are written in natural language and therefore difficult to analyze by a machine. Like any human activity, our manual classification is subject to personal bias and subjectivity. To reduce this bias, in the future, we will ask to other researchers of our lab to classify the dataset as well, verifying and discussing possible divergences of opinion. This is important as changes in this dataset may impact our findings. 

When performing our study, we used well-commented Java projects. Since our technique heavily depends on code comments, our results may be impacted by the quantity and quality of comments in a software project. To alleviate the threat, we examined multiple projects. 

\everton{Other than that, there is a risk of removing self-admitted technical debt comments while filtering license comments. To mitigate this risk we do not remove comments that contains one of task-reserved words (i.e., ``todo'', ``fixme'', or ``xxx'').}  


\noindent \textbf{External validity} consider the generalization of our findings. All of our findings were derived from comments in open source projects. To minimize external validity, we chose open source projects from different domains. That said, our results may not generalize to other open source or commercial projects. In particular, our results may not generalize to projects that have a low number or no comments. Other than that, we only analyze projects written in Java, therefore the results obtained may not generalize to projects written in other languages.