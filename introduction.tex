The software development process is filled with challenges. There are short deadlines, complex changes that need to be made, high quality expectations and an ever changing environment. Often there is much more that needs to be done than time to accomplish it. These conditions puts developers under increasing pressure to implement their tasks, while achieving many conflicting constraints. In this context, some decisions are made to allow the short term development of the project at the cost of its increased maintenance effort in the future. This phenomena is know as Technical Debt \cite{Cunningham1992}. 

With the organization of the technical debt community through the managing technical debt workshop \cite{Falessi2014MTD}, recent work has focused on the detection of technical debt \cite{Potdar2014ICSME} \cite{Zazworka2013CSE}, studying the impact of technical debt \cite{Zazworka2011MTD}
and the appearance of technical debt in the form of code smells \cite{Fontana2012MTD}. Despite many efforts to detect technical debt, its detection remains a challenge~\cite{Potdar2014ICSME}. One relatively unexplored aspect of technical debt is self-admitted technical debt, that is technical debt reported in source code comments. Self-admitted technical debt refers to the situation where developers know that the current implementation is not optimal and write comments alerting the inadequacy of the solution. 

Recently, Potdar and Shihab~\cite{Potdar2014ICSME} developed an approach to identify technical debt from code comments, and through manual inspection, were able to mine 62 patterns that effectively identify self-admitted technical debt. However, their approach does not take into consideration the different types of technical debt. Understanding the different types of self-admitted technical debt is important since: 1) it helps the community understand the limitations of understanding technical debt through code comments, 2) it allows us to complement existing technical debt detection approaches and 3) it provides us with a better understanding of the developer's point of view of technical debt.

Therefore, in this paper we examine and quantify the different types of self-admitted technical debt. To do so, we extract source code comments from 5 well commented open source projects that belongs to different application domains, namely Apache Ant, Apache Jmeter, ArgoUml, Columba and JFreeChart. In total, we examined more than 166K comments. We applied a set of 4 simple filtering heuristics to remove comments that are not likely to contain self-admitted technical debt (e.g., license comments, commented source code, Javadoc comments). Finally, these filtering heuristics resulted in a dataset of 33,093 comments that the first author manually analyzed and classified into different types of self-admitted technical debt.

When classifying the code comments, we found 5 types of self-admitted technical debt which are: design debt, defect debt, documentation debt, requirement debt and test debt. Analyzing the distribution of the comments we found that the most common type of self-admitted technical debt is design debt, making up between 42\% - 84\% of all the classified comments. In addition to our findings, we contribute a rich dataset of self-admitted technical making the data used in this study publicly available. To the best of our knowledge, there is not similar data available and we believe that the dataset will encourage future research in the area of self-admitted technical providing the necessary foundation for more advanced techniques as Natural Language Processing.  

The rest of the paper is organized as follows. Section \ref{sec:related_work} presents related work. We describe our approach and setup our case study in Section \ref{sec:approach}. Section \ref{sec:results} presents the case study results. The threats to validity are presented in Section \ref{sec:threats_to_validity} and in Section \ref{sec:conclusion} concludes the paper and discusses future work. 