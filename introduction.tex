Software development imposes several challenges as short deadlines, complex solutions and low tolerance with errors in an ever changing environment. Often there is much more that needs to be done than time to accomplish it. Developers are the ones who plays the main role in this scenario, and usually is up to them to couple with such pressure when implementing their tasks. In this context, some decisions are made to allow the short term development of the project at the cost of its increased maintenance effort in the future. This phenomena is know as Technical Debt \cite{Cunningham1992}. 

The term Technical Debt has became more popular among software engineer practitioners due its communication potential \cite{Spinola2013MTD}. It is a comprehensible metaphor for managers in need to understand what before was explained trough heavy technical language. With the adoption of the metaphor more specific ways to communicate technical debt during the life cycle of a software project was developed (e.g., design debt, documentation debt, test debt) \cite{Alves2014MTD}. Now, for the software research community, is given the challenge to understand the characteristics of each different type of technical debt, advancing the state of the art in the detection and management of technical debt.

Despite the efforts addressing this problem there is one relatively unexplored aspect of technical debt that is the technical debt contained in source code comments. Sometimes, the developers knowing that the current implementation is not the optimal, write comments alerting the situation. This kind of technical debt is called self-admitted technical debt. 

Potdar \textit{et al.} \cite{Potdar2014ICSME} developed an approach to identify these comments, however their approach does not take in consideration the  in previous work.  

and technical debt types  i.\cite{Kruchten2012}.


 In the software engineering community the topic is getting more attention and studies addressing the identification and management of technical debt. \todo{citations here}

In this context, Potdar \textit{et al.} \cite{Potdar2014ICSME} developed an approach to identify technical debt contained in the source code comments. These comments containing technical debt was called  self-admitted technical debt. However, the approach to identify these comments does not take in consideration the 

 the author developed a approach to i 
The topic have a increasing interest in the software engineering research community also, and a great deal of attention is given to    


ople has done to address this problem. 
3- what are the weakness in this solution. 
4- how do our work fit in in this context
5- the results and contributions 
6- the organization of the paper. 

take a good paper and use as a inspiration to write. 


contributions with this paper:

The definition of the different types of self-admitted technical debt found in source code comments.
The quantification of each of these types of self-admitted technical debt.
Making the dataset used in this study available to the software engineering community.  