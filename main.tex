\documentclass[conference]{IEEEtran}

\usepackage{amssymb,amsmath}
\usepackage{wrapfig}
\usepackage{multirow}
\usepackage{graphicx}
\usepackage{algorithm}
\usepackage{algorithmic}
\usepackage{times}
\usepackage{cite}
\usepackage{url}
\usepackage{booktabs}
\usepackage{subfigure}
\usepackage{fancybox}
\usepackage{color}
\usepackage{array}
\usepackage{subfigure}
\usepackage{balance}
\usepackage{epstopdf}
\usepackage{array}
\usepackage[normalem]{ulem}
\usepackage{csquotes}
\usepackage{makecell}
\usepackage[bottom]{footmisc}

\newcommand{\everton}[1]{\textcolor{yellow}{{\it [Everton: #1]}}}
\newcommand{\emad}[1]{\textcolor{red}{{\it [Emad: #1]}}}
\newcommand{\todo}[1]{\colorbox{yellow}{\textbf{[#1]}}}

\newcommand{\conclusionbox}[1]{%
	\vspace{2mm}
	\framebox[0.45\textwidth][c]{%
		\parbox[b]{0.42\textwidth}{%
			{\it #1}
		}
	}
	\vspace{2mm}
}

\newcommand{\rqi}{\textbf{RQ1- What are the types of self-admitted technical debt? How frequent are the different types of self-admitted technical debt in the studied projects ?}}

\begin{document}
\title{Detecting and Quantifying Different Types of Self-Admitted Technical Debt}

\author{\IEEEauthorblockN{Everton da S. Maldonado and Emad Shihab}

\IEEEauthorblockA{Department of Computer Science and Software Engineering\\Concordia University,
Montreal, Canada\\
\url{e_silvam@encs.concordia.ca}, \url{emad.shihab@concordia.ca}}}

\maketitle

\begin{abstract} Technical Debt is a term that has been used to express non-optimal solutions during the development of software projects. These non optimal solutions are often shortcuts that allow the project to move faster in the short term, at the cost of increased maintenance in the future. To help alleviate the impact of technical debt, a number of studies focused on the detection of technical debt. More recently, our work shown that one possible source to detect technical debt is using source code comments, also referred to as self-admitted technical debt. However, what types of technical debt can be detected using source code comments remains as an open question.
	
Therefore, in this paper we examine code comments to determine the different types of technical debt. First, we propose four simple filtering heuristics to eliminate comments that are not likely to contain technical debt. Second, we read through more than 33K comments, and we find that self-admitted technical debt can be classified into five main types - design debt, defect debt, documentation debt, requirement debt and test debt. The  most common type of self-admitted technical debt is design debt, making up between 42\% to 84\% of the classified comments. Lastly, we make the classified dataset of more than 33K comments publicly available for the community as a way to encourage future research and the evolution of the technical debt landscape.	
\end{abstract}

\IEEEpeerreviewmaketitle

\section{Introduction}
\label{sec:introduction}


ople has done to address this problem. 
3- what are the weakness in this solution. 
4- how do our work fit in in this context
5- the results and contributions 
6- the organization of the paper. 

take a good paper and use as a inspiration to write. 


contributions with this paper:

The definition of the different types of self-admitted technical debt found in source code comments.
The quantification of each of these types of self-admitted technical debt.
Making the dataset used in this study available to the software engineering community.  

\section{Related Work}
\label{sec:related_work}

\subsection{Comment related work}

Other work used code comments to understand developer tasks. For example. Storey \textit{et al.}~\cite{Storey2008ICSE} analyzed how task annotations (e.g., TODO, FIXME) play a role in improving team articulation and communication. The work closest to ours is the work by Potdar and Shihab~\cite{Potdar2014ICSME}, where code comments were used to identify technical debt. 

Similar to some of the prior work. we also use source code comments to identify technical debt. However, our main focus is on the detection of . As we have shown, our approach yield different and better results in detection . Furthermore, we propose comment patterns, that are derived from source code comments, to detect .

\subsection{Technical debt related work}

A number of studies have focused on the study of, detection and management of technical debt. Much of this work has been driven by the Managing Technical Debt Workshop effort. Fore example, Seaman \textit{et al.}~\cite{Seaman2011}, Kruchten \textit{et al.}~\cite{Kruchten2013IWMTD} and Brown \textit{et al.}~\cite{Brown2010MTD} make several reflections about the term technical debt and how it has been used to communicate the issues that developers find in the code in a way that managers can understand. Alves \textit{et al.} \cite{Alves2014MTD} complement this work by proposing a ontology on technical debt terms. Given the fact that this is a recent research area with publication dating only since 2010. In their work they gathered definitions and indicators of technical debt that was scattered across the literature, and as a result their ontology provides several different types of technical debt (e.g., architecture debt, build debt, code debt, design debt, defect debt, etc) grouped by their nature (i.e., the factor that lead to the introduction of the debt at the first place).  

Other work focused on the detection of technical debt. Zazworka \textit{et al.} \cite{Zazworka2013CSE} conducted an experiment to compare the efficiency of automated tools in comparison with human elicitation regarding the detection of technical debt. They found that there is small overlap between the two approaches, and thus it is better to combine them than replace one with the other. In addition, they concluded that automated tools are more efficient in finding defect debt, whereas developers can realize more abstract categories of technical debt.

In follow on work, Zazworka \textit{et al.}~\cite{Zazworka2011MTD} conducted a study to measure the impact of technical debt on software quality. They focused on a particular kind of design debt, namely God Classes. They found that God Classes are more likely to change, and therefore, have a higher impact in software quality. Fontana \textit{et al.}~\cite{Fontana2012MTD} investigated design technical debt appearing in the form of code smells. They used metrics to find three different code smells, namely God Classes, Data Classes and Duplicated Code. They proposed an approach to classify which one of the different code smells should be addressed first, based on a risk scale. Also related here, Potdar and Shihab~\cite{Potdar2014ICSME} used code comments to detect technical debt.They extracted the comments of four projects and analyzed more than 101,762 comments to come up with 62  patterns that indicates self-admitted technical debt. Their findings show that 2.4\% - 31\% of the files in a project contain self-admitted technical debt.

%Our work is different from the work that uses code smells to detect design technical debt since we use code comments to detect design technical debt. Also, our focus is on \emph{self-admitted} design technical debt. As we have shown in the discussion section, there is very little overlap between the  that our approach detects and the design technical debt detected using code smells (in particular God classes)

%The above work provide the necessary background to complement previous work of potdar.

\todo{add comparison between works and prune it better}

\section{Approach}
\label{sec:approach}
% \begin{figure*}[thb!]
%   \caption{Approach overview}
%   \centering
%   \label{fig:approach}
%   \includegraphics[width=1\textwidth]{figures/Approach2}
% \end{figure*}

\todo{create figure of approach}
The main goal of our study is to identify and quantify the different types of self-admitted technical debt found in source code comments. Figure I shows an overview of our approach. The following subsections detail each step of our approach.

\subsection{Data Extraction} % (fold)
\label{sub:data_extraction}

To perform our study, we obtain the source of four open source projects, namely Apache Ant, Apache Jmeter, ArgoUML and JFreeChart. We chose the aforementioned projects, since they belong to different domains, and vary in size (e.g., LOC), and in the number of contributors.

Table \ref{tab:project_details} provides statistics about each of the projects used in our study. In total, we obtained more than 132,501 comments, found in 6,330 files. We also include the release used, the number of classes, and the total lines of code (LOC). In our study, we only use the Java files to calculate the LOC. It is important to notice that the number of comments shown for each project does not represent the number of commented lines, but rather the number of individual line, block, and Javadoc comments.
/todo{complement details table with more information about the project, put references from where I took the number of developers}
% subsection data_extraction (end)
 
\subsection{Parse Source Code} % (fold)
\label{sub:parse_source_code}
After obtaining the source code of all projects, we extract the comments from their source code. We use JDeodorant \cite{Tsantalis2008CSMR}, an open-source Eclipse plug-in, to parse the source code and extract the code comments. Once extracted, we store all comments in a relational database to facilitate the processing of the data.
% subsection parse_source_code (end) 

\subsection{Filter Comments} % (fold)
\label{sub:filter_comments}

Source code comments can be used for different purposes in a project like giving context, as part of documentation, to express thoughts, opinions and authorship, and in some cases, to remove source code from the program. Comments are used freely for developers and with few formalities, if any at all. This informal environment allows developers to bring to light opinions, insights and even confessions (i.e, a poorly implemented piece of code). 

As shown in a previous study by Potdar \textit{et al.} \cite{Potdar2014ICSME}, part of these comments can be identified as technical debt. However, these comments are not the majority in a project. To tackle this problem, we develop and apply 4 filtering heuristics to narrow down the comments eliminating the ones that are less likely to be classified as technical debt.

We apply first a heuristic to remove license comments. When license comments are added to the Java files in a project they are generally placed in the first lines of the file, before the class declaration. Based on this knowledge we created a heuristic that eliminates comments that are placed before the class declaration. 

Second, a heuristic to merge multiple line comments. Some times developers make long comments, using multiple single-line comments instead of a Block comment. Treating every single line of a long comment as an individual comment causes us to miss important context details that could be recovered by treating all single-line comments as a single block comment. Therefore, we create this heuristic that searches for consecutive single-line comments and groups them.

Third, a heuristic to remove commented source code. Commented source code can be found for several different reasons. One of the possibilities could be that the code is not being currently used, or if the particular piece of code is used to debug the program. Since commented code does not have Self-admitted Design Technical Debt, we remove commented source code using a regular expressions that captures typical Java code structures.

Fourth, a heuristic to remove Javadoc comments. The Javadoc comments contain information about the purpose and use of methods and classes. That said, Javadoc comments rarely mention Self-admitted Design Technical Debt. Therefore, we create a heuristic that removes Javadoc comments. To mitigate the risk of eliminating some correct cases, we added one exception - if the comment contains one of the task-reserved words (e.g. “todo”, “fixme”, or “xxx”) we keep that Javadoc comment.

The steps mentioned above significantly reduced the number of comments in our dataset and helped us focus on the most applicable and insightful comments. For example, in the Apache Ant project, applying the above steps helped reduce the number of comments from 21,587 to 4,140 comments.

% subsection filter_comments (end)

\subsection{Manual Classification} % (fold)
\label{sub:manual_classification}

To manually classify the comments we first developed a Java based tool that show one comment at time and gives a list of possible classifications that will be assigned to the comment. Using this tool we were able to classify more than \todo{add number of classified comments} comments. 
% into one of following categories of technical debt: Documentation Debt, Design Debt, Defect Debt, Implementation Debt or Text  Debt.  

% subsection manual_classification (end)

\begin{table*}[!hbt]
      \begin{center}
            \caption{Project Details}
            \label{tab:project_details}
            \begin{tabular}{l| c c c c }
            \toprule
            \textbf{Project}   & \textbf{Release}          & \textbf{LOC}     & \textbf{\# of comments} & \textbf{\# of contributors} \\ \midrule 
              Apache Ant       & 1.7.0                     & 115,881          & 21,587                                & 70  \\                                   
              Apache Jmeter    & 2.10                      & 81,307           & 20,084                                & 32  \\                                   
              ArgoUML          & 0.34                      & 176,839          & 67,716                                & 87  \\                                   
              Columba          & 1.4                       & 100,200          & 33,895                                & 9   \\                                   
              JFreeChart       & 1.0.19                    & 132,296          & 23,474                                & 18  \\ \bottomrule
            \end{tabular}
      \end{center}
\end{table*}


\section{Case Study Results}
\label{sec:results}
The goal of our study is to understand technical debt found in source code comments. To do so, we first manually analyze the comments identifying technical debt comments in the selected projects. We divide our research question in two parts first, we classify the comments into technical debt types accordingly with their nature. Second, we quantify these comments identifying the most common types of self-admitted technical debt. In the remainder of this section we detail the motivation, approach and results for our research question.  

\vspace{3mm}
\noindent\rqi
\vspace{3mm}

\noindent\textbf{Motivation:} As shown in previous work \cite{Potdar2014ICSME}, self-admitted technical can be an indicator of non-optimal solutions. However, technical debt is a general term, and there are many different types of technical debt \cite{Alves2014MTD}. Although we know that self-admitted technical exists, the different kinds of self-admitted technical debt is still  unknown. Answering this question is important as different types of debt has different approaches to be solved, and therefore each different type may need a tailored solution. 

\vspace{1mm}
\noindent\textbf{Approach:} To identify the different types of debt found in the comments we manually read trough all filtered comments as described in Section \ref{sec:approach}. While examining the comments we classify each comment by the nature of the debt as proposed by Alves \textit{et al.} in previous work. 

During the classification we notice that some comments can be classified in more than one type of debt (e.g., A comment reporting a Design Debt can also means that a particular piece of code is causing an unexpected behavior, which is a Defect Debt). Although this is an ambiguous situation, and may have different interpretations depending of who is reading the comments, we defined that each comment would have just one classification type for the sake of clarity. To mitigate the chance of misclassifying these comments, we take in consideration the more meaningful type for each comment in a given scenario. To do so, a more detail investigation was necessary (i.e., looking at the source code). In total we read and classified 33,093 comments from five open source projects. The classification took approximately 85 hours and was performed by the first author of the paper. 

\vspace{1mm}
\noindent\textbf{Results:} First, we present the types of self-admitted technical debt that we could find in the source code comments that we classified. 

\begin{itemize}
  \item \textbf{Self-admitted Design debt:} 
  These comments indicates that there is a problem with the design of the code. They can be comments about misplaced code, lack of abstraction, long methods, poor implementation, workarounds and temporary solution were classified in this type. Lets consider the following comments:
  
  \vspace{1mm}
  \begin{displayquote}
     \textit{``TODO: - This method is too complex, lets break it up''}
     
     \vspace{1mm}

     \textit{``/* TODO: really should be a separate class */''}
  \end{displayquote}
  \vspace{1mm}

  These comments are clear examples of what we consider a self-admitted design debt. The authors states their analysis and what need to be done in order to improve the current design of the code. Although, during the analysis we came across more challenging comments that expressed design problems in an indirect way:  
  
  \vspace{1mm}
  \begin{displayquote}
     \textit{``// I hate this so much even before I start writing it. // Re-initialising a global in a place where no-one will see it just // feels wrong.  Oh well, here goes.''}

     \vspace{1mm}

     \textit{``//quick \& dirty, to make nested mapped p-sets work:''}
  \end{displayquote}
  \vspace{1mm}

  In the above examples the authors are certain to be implementing code that does not represent the best solution. Intuitively, we know that kind of implementation will degrade the design of the code and should be avoided. 

  \vspace{1mm}
  \begin{displayquote}
      \textit{``// probably not the best choice, but it solves the problem of // relative paths in CLASSPATH''}

      \vspace{1mm}

      \textit{``//I can't get my head around this; is encoding treatment needed here?''}
  \end{displayquote}
  \vspace{1mm}

  The above comments expressed doubt and uncertainty when implementing the code and were considered as self-admitted design debt as well.

  \item \textbf{Self-admitted Defect debt:} In defect debt comments the author states that a particular piece of code do not have the expected behavior, meaning that there is a defect in the code. 
  
  \vspace{1mm}
  \begin{displayquote}
      \textit{``// Bug in above method''}

      \vspace{1mm}

      \textit{``// WARNING: the OutputStream version of this doesn't work!''}
  \end{displayquote}
  \vspace{1mm}
  
  As shown in these examples there are defects that are known by the developers, but for some reason is not fixed yet. 

  \item \textbf{Documentation debt:} In the documentation debt comments the author express that there is not proper documentation supporting that part of the program.
  
  \vspace{1mm}
  \begin{displayquote}
  	\textit{``**FIXME** This function needs documentation''}
  	
  	\vspace{1mm}
  	
  	\textit{``// TODO Document the reason for this''}
  \end{displayquote}
  \vspace{1mm}
  
  \item \textbf{Self-admitted Requirement debt:} Requirement debt comments express incompleteness of the method, class or program.
  
  \begin{displayquote}
  	\textit{``/TODO no methods yet for getClassname''}
  	
  	\vspace{1mm}
  	
  	\textit{``//TODO no method for newInstance using a reverse-classloader''}

  	\vspace{1mm}
  	
  	\textit{``TODO: The copy function is not yet * completely implemented - so we will  * have some exceptions here and there.*/''}  
  	
  \end{displayquote}
  \vspace{1mm}  
  
  	The last example shows a comment that could be considered as having more than one type of debt. (i.e., requirement debt and defect debt), but as mentioned in the classification approach, we choose to maintain one type only for each comment. 
  	
  	In our understanding, the defect debt expressed in the comment would not exist if the requirement debt did not exists. Therefore, the main debt in this comment is a requirement debt (i.e., incomplete implementation of the copy function). 
  	
  	One more reason that we give a comment only one classification type is that there is no way to tell if the other requirement debts are not causing an unexpected behavior, except in the case that the author of the comment express is. 
  
  \vspace{1mm}
  \item \textbf{Self-admitted Test debt:} Test debt comments are the ones that express the need for implementation or improvement of the current tests. As shown in the examples below, test debt comments are very straight forward in their meaning. 
  
  \begin{displayquote}
  	\textit{``// TODO - need a lot more tests''}
  	
  	\vspace{1mm}
  	
  	\textit{``//TODO enable some proper tests!!''}
  \end{displayquote}
  \vspace{1mm}  
    
\end{itemize}

After classifying the comments, we notice that not all of the types mentioned in \cite{Alves2014MTD} could be found. We argue that some of the types like People Debt or Infrastructure Debt is really less probable to appear in source code comments, whereas other types as Build Debt could not be found because we are examining just comments found in Java classes, therefore not taking in consideration build scripts that can be written in other languages (e.g., Maven and Ant use xml files as build scripts). 

Other than that, types like Test Automation debt and Test Debt were considered as the same type in our study due to te resemblance of their purpose. In a similar way, Process Debt, Service Debt, Code Debt and Architecture Debt were considered as Self-admitted design debt in our study.

\conclusionbox{We found 5 different types of self-admitted technical debt in the comments of the studied projects. Design Debt, Defect Debt, Documentation Debt, Requirement Debt and Test Debt. Althoug there are comments that can have more the one type of debt, we found that there is a main type of debt that the comment can be classified to. }

% Table \ref{tab:technical_debt_project} shows the total amount of self-admitted technical debt found in the analyzed projects. For example, in Apache Ant out of the 4,140 analyzed comments 134 turnout to be classified as self-admitted technical debt, which represents 3.2\% of the comments of the analyzed comments. 

% Analyzing the results of the classification we found that the comments could be classified in one of the following types of debt: design debt, defect debt, documentation debt, requirement debt and  test debt. 
 
% To a comment be classified as design debt it has to indicate that there is a problem with design of the code (e.g., misplaced code, lack of abstraction, long method, poor implementation). Table \ref{tab:design_debt_detail} show some concrete examples of the comments classified as Design Debt. 

% In defect debt comments the author/s states that that particular piece of code do not have the expected behavior. For documentation debt comments the author/s wrote that there is not documentation supporting that part of the program. Requirement debt comments express the fact that incompleteness of the method, class or program. Finally, test debt comments are the ones that express the need for implementation or improvement of the current tests. We show examples of defect, documentation, requirement and test debt in Tables \ref{tab:defect_debt_detail}, \ref{tab:documentation_debt_detail}, \ref{tab:requeriment_debt_detail}, \ref{tab:test_debt_detail} respectively.   

%\begin{table*}[!hbt]
%      \begin{center}
%            \caption{Self-admitted technical debt per project}
%            \label{tab:technical_debt_project}
%            \begin{tabular}{l| c c c }
%            \toprule
%            \textbf{Project}      & \textbf{Analyzed comments}     & \textbf{Self-admitted TD comments} & \textbf{Percentage} \\ \midrule 
%              Apache Ant          & 4,140                          & 134                                & 3.2  \\                                   
%              Apache Jmeter       & 8,163                          & 375                                & 4.6  \\                                   
%              ArgoUML             & 9,788                          & 1,653                              & 16.8 \\                                   
%              Columba             & 6,569                          & 295                                & 4.4 \\                                   
%              JFreeChart          & 4,433                          & 219                                & 4.9  \\ \bottomrule
%            \end{tabular}
%      \end{center}
%\end{table*}

% \begin{table}[!hbt]
%       \begin{center}
%             \caption{Self-admitted design technical debt examples}
%             \label{tab:design_debt_detail}
%             \begin{tabular}{l}
%             \toprule
%             \textbf{Comments}     \\ \midrule 
%              // XXX Move to Project ( so it is shared by all helpers ) \\                                   
%              // XXX maybe use reflection to addPathElement (other patterns ?) \\                                   
%              // TODO: move this to components \\                                   
%              // Can be written better... this is too hacky! \\                                   
%              // Yuck: TIFFImageEncoder uses Error to report runtime problems \\ \bottomrule
%             \end{tabular}
%       \end{center}
% \end{table}

% \begin{table}[!hbt]
%       \begin{center}
%             \caption{Self-admitted defect technical debt examples}
%             \label{tab:defect_debt_detail}
%             \begin{tabular}{l}
%             \toprule
%             \textbf{Comments}     \\ \midrule 
%              // FIXME formatters are not thread-safe\\                                   
%              // Bug in above method \\                                   
%              // TODO: This does not work! (MVW)\\                                   
%              // WARNING: the OutputStream version of this doesn't work!\\                                   
%              // TODO: This looks backwards. Left over from issue 2034? \\ \bottomrule
%             \end{tabular}
%       \end{center}
% \end{table}

% \begin{table}[!hbt]
%       \begin{center}
%             \caption{Self-admitted documentation technical debt examples}
%             \label{tab:documentation_debt_detail}
%             \begin{tabular}{l}
%             \toprule
%             \textbf{Comments}     \\ \midrule 
%             // **FIXME** This function needs documentation\\                                   
%             // TODO Document the reason for this\\                                   
%             // TODO: Document exceptional behaviour.\\                                   
%             // TODO: centralise this knowledge.\\  \bottomrule
%             \end{tabular}
%       \end{center}
% \end{table}

% \begin{table}[!hbt]
%       \begin{center}
%             \caption{Self-admitted requirement technical debt examples}
%             \label{tab:requeriment_debt_detail}
%             \begin{tabular}{l}
%             \toprule
%             \textbf{Comments}     \\ \midrule 
%             // TODO: not implemented\\                                   
%             // TODO Auto-generated constructor stub\\                                   
%             // TODO: i18n\\                                   
%             // TODO no methods yet for getClassname  \\                                   
%             // TODO no method for newInstance using a reverse-classloader \\ \bottomrule
%             \end{tabular}
%       \end{center}
% \end{table}

% \begin{table}[!hbt]
%       \begin{center}
%             \caption{Self-admitted test technical debt examples}
%             \label{tab:test_debt_detail}
%             \begin{tabular}{l}
%             \toprule
%             \textbf{Comments}     \\ \midrule 
%             // TODO - need a lot more tests\\                                   
%             // TODO enable some proper tests!!\\                                   
%             // TODO add tests for SaveGraphics\\                                   
%             // TODO these assertions should be separate tests\\ \bottomrule
%             \end{tabular}
%       \end{center}
% \end{table}

Second, we quantify and analyze the distribution of the self-admitted technical debt in the comments. Figure \todo{add figure} shows the percentage of each type of self-admitted technical debt.


the number of requirement debt shows the maturity of the project by the perspective of the developers. a lot of opportunities to improve features of the project can mean that the project still in a immature phase. or that there still a lot of room to improvement. it means that is not in the more stable state that it can be. or that the authors were expecting. when the project is new there is a lot of thing that the developers still want to develop. in more stable projects the predominance is design debits not requirements (implementation). the number of features that one product offers can play a role in the implementation debt as well, as the role of the application is more complex one will find more things to implement as well. like argo has many features, and many users, this can provide the developers with the necessary direction to evolve the project . whereas columba has just a few developers so there are a lot of things that need to be implemented as well, the number of users can put this number up or down , a lot of users can force the developers to better implement the software. tell about the the difference between very large softwares against very specialist systems. 


% contributions with this paper:

% The definition of the different types of self-admitted technical debt found in source code comments.
% The quantification of each of these types of self-admitted technical debt.
% Making the dataset used in this study available to the software engineering community.  

%\begin{table*}[!hbt]
%      \begin{center}
%            \caption{Self-Admitted Technical Debt distribution}
%            \label{tab:td_distribution}
%            \begin{tabular}{l| c c c c c}
%            	\toprule
%            	\textbf{Project} & \textbf{\# Defect comments} & \textbf{\# Design comments} & \textbf{\# Documentation comments} & \textbf{\# Implementation comments} & \textbf{\# Test comments} \\ \midrule
%            	Apache Ant       & 13                          & 95                          & 0                                  & 16                                  & 16                        \\
%            	Apache Jmeter    & 22                          & 316                         & 3                                  & 22                                  & 12                        \\
%            	ArgoUML          & 127                         & 801                         & 30                                 & 651                                 & 44                        \\
%            	Columba          & 13                          & 126                         & 16                                 &                                     & 6                         \\
%            	JFreeChart       & 9                           & 184                         & 0                                  & 25                                  & 1                         \\ \bottomrule
%            \end{tabular}
%      \end{center}
%\end{table*}
%
% \begin{table*}[!hbt]
%       \begin{center}
%             \caption{Apache Ant Self-Admitted Technical Debt distribution}
%             \label{tab:ant_td_details}
%             \begin{tabular}{l| c c }
%             	\toprule
%             	\textbf{Type}  & \textbf{\# of comments} & \textbf{Percentage} \\ \midrule
%             	Defect         & 13                      & 9.70                \\
%             	Design         & 95                      & 70.89               \\
%             	Documentation  & 0                       & 0.00                \\
%             	Implementation & 16                      & 11.94               \\
%             	Test           & 10                      & 7.46                \\ \bottomrule
%             \end{tabular}
%       \end{center}
% \end{table*}
%
% \begin{table*}[!hbt]
%       \begin{center}
%             \caption{Apache Jmeter Self-Admitted Technical Debt distribution}
%             \label{tab:jmeter_td_details}
%             \begin{tabular}{l| c c }
%             	\toprule
%             	\textbf{Type}  & \textbf{\# of comments} & \textbf{Percentage} \\ \midrule
%             	Defect         & 22                      & 5.86                \\
%             	Design         & 316                     & 84.26               \\
%             	Documentation  & 3                       & 0.8                 \\
%             	Implementation & 22                      & 5.86                \\
%             	Test           & 12                      & 3.2                 \\ \bottomrule
%             \end{tabular}
%       \end{center}
% \end{table*}
%
% \begin{table*}[!hbt]
%       \begin{center}
%             \caption{ArgoUml Self-Admitted Technical Debt distribution}
%             \label{tab:argo_td_details}
%             \begin{tabular}{l| c c }
%             	\toprule
%             	\textbf{Type}  & \textbf{\# of comments} & \textbf{Percentage} \\ \midrule
%             	Defect         & 127                     & 7.68                \\
%             	Design         & 801                     & 48.45               \\
%             	Documentation  & 30                      & 1.81                \\
%             	Implementation & 651                     & 39.38               \\
%             	Test           & 44                      & 2.66                \\ \bottomrule
%             \end{tabular}
%       \end{center}
% \end{table*}
%
% \begin{table*}[!hbt]
%       \begin{center}
%             \caption{JFreechart Self-Admitted Technical Debt distribution}
%             \label{tab:jfreechart_td_details}
%             \begin{tabular}{l| c c }
%             	\toprule
%             	\textbf{Type}  & \textbf{\# of comments} & \textbf{Percentage} \\ \midrule
%             	Defect         & 9                       & 4.10                \\
%             	Design         & 184                     & 84.01               \\
%             	Documentation  & 0                       & 0.0                 \\
%             	Implementation & 25                      & 11.41               \\
%             	Test           & 1                       & 0.45                \\ \bottomrule
%             \end{tabular}
%       \end{center}
% \end{table*}
%
%\begin{table*}[!hbt]
%      \begin{center}
%            \caption{Columba Self-Admitted Technical Debt distribution}
%            \label{tab:jfreechart_td_details}
%            \begin{tabular}{l| c c }
%            	\toprule
%            	\textbf{Type}  & \textbf{\# of comments} & \textbf{Percentage} \\ \midrule
%            	Defect         & 13                      & 4.10                \\
%            	Design         & 126                     & 84.01               \\
%            	Documentation  & 16                      & 0.0                 \\
%            	Implementation & 134                     & 11.41               \\
%            	Test           & 6                       & 0.45                \\ \bottomrule
%            \end{tabular}
%      \end{center}
%\end{table*}


\section{Threats to validity}
\label{sec:threats_to_validity}
\noindent\textbf{Internal validity} consider the relationship between theory and observation, in case the measured variables do not measure the actual factors. To classify the source code comments we heavily depended on manual classification due the fact that comments are written in natural language and therefore needed to be interpreted by a human. Like any human activity, our manual classification is subject to personal bias and subjectivity. To reduce this bias in the future we will ask to other researchers of our lab to classify the dataset, verifying and discussing possible divergences of opinion. Changes in this dataset may impact our findings. When performing our study, we used well-commented Java projects. Since our technique heavily depends on code comments, our results and performance measures may be impacted by the quantity and quality of comments in a software project.  

\noindent \textbf{External validity} consider the generalization of our findings. All of our findings were derived from comments in open source projects. To minimize external validity, we chose open source projects from different domains. That said, our results may not generalize to other open source or commercial projects. In particular, our results may not generalize to projects that have a low number or no comments. Other than that, we only analyze projects written in Java, the results obtained may not generalize to projects written in other languages.

\section{Conclusion and Future work}
\label{sec:conclusion}
The term technical debt is being used for practitioners and researchers in the software engineer community to express shortcuts and workarounds employed in software projects. These shortcuts will most often impact the maintainability of the project hindering the development if not addressed properly. Our work explore specifically self-admitted technical debt, that is the technical debt deliberately introduced by the developers and reported through source code comments.

In our study we analyzed the comments of 5 open source projects which are Apache Ant, Apache Jmeter, ArgoUml , Columba and JFreeChart. These projects are considered well commented and they belong to different application domains. We used them to understand the characteristics of self-admitted technical debt types creating a rich dataset with more than 33,093 classified comments.

%In our approach we present 4 simple filtering heuristics that made feasible the hard task to manually classify the comments of a project. These filtering heuristics significantly reduced the number of comments filtering approximately 81\% of the 166,756 extracted comments. This allowed the classification to be focused on the most applicable and insightful comments which was 33,093 comments in total. 

We find that self-admitted technical debt can be classified into five types: design debt, defect debt, documentation debt, requirement debt and test debt. We also provide concrete examples of each one of the mentioned types and the rationale to classify them as it was. Moreover, we find that the majority of the self-admitted technical debt comments are design debt. Design debt ranged from 42\% to 84\% across the projects. The second most frequent type was requirement debt ranging from 5\% to 45\%. Based on this result, we can say that the self-admitted technical debt types that developers admit to the most are related with the design of the project, potentially indicating that developers feel the need to admit and be forthcoming about such debt. Examining the reasons for these types of debt is an interesting future direction that we plan to pursue.

Other contribution of our study is that we make publicly available the resulting dataset of our classification \todo{add link}. We hope that this will encourage future research in the area of self-admitted technical debt as, to the best of our knowledge, this is the first dataset of this kind. We also think that the information provided by this dataset can be a cornerstone for more advanced techniques as natural language processing.   

In a future work we plan to improve the current classification adding more projects to it. With a richer dataset we expect that more patterns and characteristics of the self-admitted technical types will be retrieved. We also plan to use this database to mine unique sequential patterns, an advanced technique of natural language processing, which may lead to more automated ways to identify self-admitted technical debt. 


\bibliographystyle{IEEEtran}
\balance
\bibliography{bib}

\end{document}
